
\documentclass{article}
\usepackage{minx_math}
\usepackage{bbm}

\title{Miscellaneous Notes for Elliptical Log-Concave Density}

\begin{document}
\maketitle


Define $K_n := K \backslash (1 - \frac{1}{n}) K$. Let $Z \sim \mathrm{Unif}(K)$, then we have that
\[
  Z = Z_n W + Z_n' (1 - W)
\]

where $W := \mathbbm{I}\{Z \in K_n\}$, $Z_n \sim \mathrm{Unif}(K_n)$, $Z_n' \sim \mathrm{Unif}((1 - \frac{1}{n}) K)$, and $W, Z_n, Z'_n$ are independent. If $W = 1$, then $Z/\|Z\|_K = Z_n / \| Z_n \|_K$ and if $W = 0$, then $Z/\|Z\|_K = Z'_n / \| Z'_n \|_K$. Thus,
\[
  \frac{Z}{\|Z\|_K} = \frac{Z_n}{\|Z_n\|_K} W + \frac{Z_n'}{\| Z'_n \|_K} (1 - W).
\]
Note that $\frac{Z_n'}{\| Z'_n \|_K}$ is identically distributed as $\frac{Z}{\|Z\|_K}$ since $Z_n'$ is identically distributed as $(1 - \frac{1}{n}) Z$. Then,

\[
  \phi_{\frac{Z}{\|Z\|_K}}(t) = \lambda \phi_{\frac{Z}{\|Z\|_K}}(t) + (1 - \lambda) \phi_{\frac{Z_n}{\|Z_n\|_K}}(t),
\]
where $\lambda = \mathbb{P}(W = 1)$. Thus, we have that $\frac{Z_n}{\| Z_n \|_K}$ is identically distributed as $\frac{Z}{\|Z\|_K}$.

Since $\| Z_n - \frac{Z_n}{\|Z_n\|_K} \| \leq \frac{1}{\|Z_n \|_K} - 1 \leq \frac{1}{1 - \frac{1}{n}} - 1 \rightarrow 0$, we have that $Z_n$ converges weakly to $\frac{Z}{\|Z\|_K}$. 



\section{Envelope Search}

\textbf{Problem:} Let $h(r) = r^{p-1} g(r) c_p$ be a density where $g(r)$ is log-concave and decreasing. Suppose also that the second moment is $p$. 
\begin{align*}
\int r^{p-1} g(r) c_p dr &= 1 \\
\int r^2 r^{p-1} g(r) c_p dr & = p
\end{align*}
Let $\mathcal{H}$ be the set of all such densities. Then, we want to have an exponentially decaying envelope

\[
\sup_{h \in \mathcal{H}} h(r) \leq \exp( - a_p r + b_p )
\]

for scalar $a_p, b_p$ dependent on $p$.

\subsection{Thoughts and Examples}

One possibly useful fact. If $f(x)$ is an isotropic log-concave density, then there exists absolute constants $a,b$ such that $f(x) \leq \exp( - a x + b)$. 

Therefore, if $f( \frac{x}{\sigma} ) \frac{1}{\sigma}$ has variance $\sigma^2$ and can be bounded by $\exp( - \frac{a}{\sigma} x + b - \log \sigma)$. \\




One example to keep in mind is if $g(r) = M > 0$ is uniform on $[0, r_0]$ and $0$ elsewhere. It is easy to solve for $r_0$:

\begin{align*}
\int_0^{r_0} r^{p-1} M c_p dr = \frac{r_0^p}{p} M c_p = 1
\end{align*}

\[
\int_0^{r_0} r^{p+1} M c_p dr = \frac{r_0^{p+2}}{p+2} M c_p = p 
\]

\begin{align*}
\frac{r_0^{p+2}}{p+2} M c_p = \frac{r_0^p}{p} M c_p \frac{p}{p+2} r_0^2 = \frac{p}{p+2} r_0^2 = p
\end{align*}

Therefore, $r_0 = \sqrt{p+2}$. This density has vanishing variance and its maximum value explodes. The maximum value is $h(r_0) = r_0^{p-1} M c_p = \frac{p}{r_0} \frac{r_0^{p-1}}{p} M c_p = \frac{p}{\sqrt{p+2}}$. 

To compute the variance, we first find the mean.
\begin{align*}
\int_0^{r_0} r^p M c_p dr = \frac{r_0^{p+1}}{p+1} M c_p = \frac{p}{p+1} \sqrt{p+2} 
\end{align*}
\[
\textrm{variance: } \E Y^2 - (\E Y)^2 = p^2 - \left(\frac{p}{p+1} \right)^2 (p+2) = \frac{p}{(p+2)^2}
\]


\textbf{Two points bound.} Let $M = \log g(0)$. Let $r_0 > 0$ and let $M' = \log g(r_0)$. Define $\Delta = M - M' \geq 0$. 

Then, we have the following upper and lower bound on $g$:

\begin{align*}
\log g(r) \geq \left\{ \begin{array}{cc} M - \Delta \frac{r}{r_0}   & r \leq r_0 \\
                             -\infty & r \geq r_0 \end{array} \right\}
\end{align*}

\begin{align*}
\log g(r) \leq \left\{ \begin{array}{cc} M & r \leq r_0 \\
                         M - \Delta \frac{r}{r_0} & r \geq r_0 
                      \end{array} \right\}
\end{align*}

So then, we have that
\begin{align*}
1 &= \int_0^\infty r^{p-1} g(r) c_p dr \\
 &\geq  \int_0^{r_0} r^{p-1} \exp\left( M - \Delta \frac{r}{r_0} \right) c_p dr \\
\end{align*}
\subsection{Change of Variables}

Recall that the density we are interested in is:
\[
h(r) = r^{p-1} g(r) c_p
\]

Satisfying the two conditions that
\begin{align*}
\int_0^\infty r^{p-1} g(r) c_p dr &= 1 \\
\int_0^\infty r^{p+1} g(r) c_p dr &= p 
\end{align*}

Let us perform a change of variables: $s = \frac{r}{\sqrt{p}}$ and thus $r = s \sqrt{p}$. 

Then, the two integral equations become:
\begin{align*}
c_p \sqrt{p}^p \int_0^\infty s^{p-1} g(\sqrt{p}(s)) ds &= 1 \\
c_p \sqrt{p}^{p+2} \int_0^\infty s^{p+1} g(\sqrt{p}(s)) ds &= p 
\end{align*}

With some cancelation and with the replacement of $\tilde{g}(s) = c_p \sqrt{p}^p g(\sqrt{p}(s))$, we have that

\begin{align*}
\int_0^\infty s^{p-1} \tilde{g}(s) ds &= 1 \\
\int_0^\infty s^{p+1} \tilde{g}(s) ds &= 1
\end{align*}

Note that $\tilde{g}(s)$ is log-concave and decreasing. 

\textbf{An Observation}

Let $r_0$ be arbitrary. Then we have that

\begin{align*}
\int_0^{r_0} s^{p-1} (1 - s^2) \tilde{g}(s) ds + \int_{r_0}^\infty s^{p-1} (1 - s^2) \tilde{g}(s) ds = 0
\end{align*}

If $r_0 \leq 1$, then the first term is positive, which implies that the second term is negative.
If $r_0 \geq 1$, then the second term is negative, which implies that the first term is positive.

Thus, for any $r_0$, we have that the first term is positive and the second term is negative. 

\subsection{Hinge Example}


The analysis of this example provides some useful calculations.

Let $g(r)$ be of the form:
\[
g(r) = \left\{ \begin{array}{ll}
       e^{m_0} & r \leq r_0 \sqrt{p} \\
    e^{m_0 - a(r - r_0\sqrt{p})} & r \geq r_0 \sqrt{p} 
          \end{array} \right.
\]

$g$ is thus parametrized by three parameters: $m_0, a, r_0$.

We want $g(r)$ to satisfy two integral conditions:

\begin{align*}
\int_0^\infty r^{p-1} g(r) c_p dr &= 1 \\
\int_0^\infty r^{p+1} g(r) c_p dr &= p 
\end{align*}

The first integral equation breaks down into two halves:
\begin{align*}
\int_0^{r_0 \sqrt{p}} r^{p-1} e^{m_0} c_p dr + \int_{r_0 \sqrt{p}}^\infty r^{p-1} e^{m_0} e^{-a ( r - r_0 \sqrt{p} )} c_p dr = 1
\end{align*}

We apply a change of variables: $s = \frac{r}{\sqrt{p}}$ and $r = s\sqrt{p}$. 

\begin{align*}
e^{m_0} c_p \sqrt{p}^p \left\{
   \int_0^{r_0} s^{p-1} ds + \int_{r_0}^\infty s^{p-1} e^{- a \sqrt{p}(s - r_0)}  ds  \right\} = 1
\end{align*}
Likewise, we have that second equation as well:
\begin{align*}
e^{m_0} c_p \sqrt{p}^{p+2} \left\{
   \int_0^{r_0} s^{p+1} ds + \int_{r_0}^\infty s^{p+1} e^{- a \sqrt{p}(s - r_0)}  ds  \right\} = p
\end{align*}

We will simplify by letting $\bar{a} = a \sqrt{p}$. Then, we have:

\begin{align*}
e^{m_0} c_p \sqrt{p}^p \left\{
   \int_0^{r_0} s^{p-1} ds + \int_{r_0}^\infty s^{p-1} e^{- \bar{a} (s - r_0)}  ds  \right\} &= 1 \\
e^{m_0} c_p \sqrt{p}^p \left\{
   \int_0^{r_0} s^{p+1} ds + \int_{r_0}^\infty s^{p+1} e^{- \bar{a} (s - r_0)}  ds  \right\} &= 1
\end{align*}

Setting the two equation equal to each other:
\begin{align*}
\int_0^{r_0} s^{p-1} ds - \int_0^{r_0} s^{p+1} ds + 
        \int_{r_0}^\infty s^{p-1} e^{-\bar{a}(s - r_0)} ds - 
        \int_{r_0}^{\infty} s^{p+1} e^{- \bar{a} (s - r_0)} ds &= 0 \\
\left( \frac{r_0^p}{p} - \frac{r_0^{p+2}}{p+2} \right) + 
        \int_{r_0}^\infty s^{p-1} e^{-\bar{a}(s - r_0)} ds - 
        \int_{r_0}^{\infty} s^{p+1} e^{- \bar{a} (s - r_0)} ds &= 0 \\
\end{align*}

We know that for all plausible $r_0$, it must be that $\frac{r_0^P}{p} - \frac{r_0^{p+2}}{p+2} \geq 0$. Thus, 
to solve for the maximum value of $r_0$, we set $\frac{r_0^p}{p} = \frac{r_0^{p+2}}{p+2}$, yielding
$\sqrt{1 + \frac{2}{p}}$. 

Now we turn our attention to $\bar{a}$. At $r_0 = 1$, we have that

\begin{align*}
\int_{1}^\infty s^{p-1}(1 - s^2) e^{- \bar{a}(s - 1)} &= \frac{1}{p+2} - \frac{1}{p} \\
\int_{1}^\infty s^{p-1}(1 - s^2) e^{- \bar{a} s} ds &= e^{- \bar{a}} \left( \frac{1}{p+2} - \frac{1}{p} \right) \\
\int_{1}^\infty s^{p-1}  e^{- \bar{a} s} - s^{p+1} e^{ - \bar{a}s} ds &= e^{- \bar{a}} \left( \frac{1}{p+2} - \frac{1}{p} \right) \\
\end{align*}




\end{document}